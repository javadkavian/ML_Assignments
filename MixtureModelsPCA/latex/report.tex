\documentclass{article}
\usepackage{graphicx}
\usepackage{listings}
\usepackage{xcolor}
\usepackage{tcolorbox}
\usepackage{tikz}
\usepackage{amsmath}


\lstdefinestyle{mystyle}{
    language=Python,
    basicstyle=\small\ttfamily,
    keywordstyle=\color{blue},
    stringstyle=\color{red},
    commentstyle=\color{green},
    breaklines=true,
    showstringspaces=false,
    frame=tb,
    backgroundcolor=\color{gray!10},
}

\title{ML\_HW5}
\author{Mohammad Javad Pesarakloo\\810100103}
\date{\today}

\begin{document}
\maketitle

\section*{Question 1}
\subsection*{A}
We denote mean of class one with $m_1$ and mean of class two with $m_2$.First we need to find $m_1$ and $m_2$:
\begin{equation*}
    m_1 = \begin{pmatrix}
        3\\6
    \end{pmatrix}, m_2 = \begin{pmatrix}
        8.4\\7.6
    \end{pmatrix}
\end{equation*}
Now we subtract means from their corresponding samples:
\begin{equation*}
    X_1 - m_1 = \lbrace \begin{pmatrix}
        1\\-2.6
    \end{pmatrix},\begin{pmatrix}
        -1\\0.4
    \end{pmatrix},\begin{pmatrix}
        -1\\-0.6
    \end{pmatrix},\begin{pmatrix}
        0\\2.4
    \end{pmatrix},\begin{pmatrix}
        1\\0.4
    \end{pmatrix} \rbrace
\end{equation*}

\begin{equation*}
    X_2 - m_2 = \lbrace \begin{pmatrix}
        0.6\\2.4
    \end{pmatrix},\begin{pmatrix}
        -2.4\\0.4
    \end{pmatrix},\begin{pmatrix}
        0.6\\-2.6
    \end{pmatrix},\begin{pmatrix}
        -0.4\\-0.6
    \end{pmatrix},\begin{pmatrix}
        1.6\\0.4
    \end{pmatrix} \rbrace
\end{equation*}

Now we can calculate $S_1$ and $S_2$:
\begin{equation*}
    S_1^2 = \begin{pmatrix}
        1 && -2.6\\-2.6 && 6.76
    \end{pmatrix} + \begin{pmatrix}
        1 && -0.4\\-0.4&&0.16
    \end{pmatrix}+\begin{pmatrix}
        1 && 0.6 \\ 0.6 && 0.36
    \end{pmatrix} + \begin{pmatrix}
        0 && 0 \\ 0 && 5.76
    \end{pmatrix} + \begin{pmatrix}
        1 && 0.4 \\ 0.4 && 0.16
    \end{pmatrix}
\end{equation*}

\begin{equation*}
    = \begin{pmatrix}
        4 && -2 \\ -2 && 13.2
    \end{pmatrix}
\end{equation*}

\begin{equation*}
    S_2^2 = \begin{pmatrix}
        0.36&&1.44\\1.44&&5.76
    \end{pmatrix} + \begin{pmatrix}
        5.76&&-0.96\\-0.96&&0.16
    \end{pmatrix} + \begin{pmatrix}
        0.36&&-1.56\\-1.56&&6.76
    \end{pmatrix} + \begin{pmatrix}
        0.16&&0.24\\0.24&&0.36
    \end{pmatrix}+\begin{pmatrix}
        2.56&&0.64\\0.64&&0.16
    \end{pmatrix}
\end{equation*}

\begin{equation*}
    = \begin{pmatrix}
        9.2 && -0.2\\-0.2&&13.2
    \end{pmatrix}
\end{equation*}

And now $S_w$ can be calculated:
\begin{equation*}
    S_w = \begin{pmatrix}
        4 && -2 \\ -2 && 13.2
    \end{pmatrix} + \begin{pmatrix}
        9.2 && -0.2\\-0.2&&13.2
    \end{pmatrix} = \begin{pmatrix}
        13.2 && -2.2 \\ -2.2 && 26.4
    \end{pmatrix}
\end{equation*}

\subsection*{B}
\begin{equation*}
    m_1 - m_2 = \begin{pmatrix}
        3 && 3.6
    \end{pmatrix} - \begin{pmatrix}
        8.4 && 7.6
    \end{pmatrix} = \begin{pmatrix}
        -5.4 && -4
    \end{pmatrix}
\end{equation*}
\begin{equation*}
    S_B = \begin{pmatrix}
        -5 && -4    
    \end{pmatrix} \begin{pmatrix}
        -5.4\\-4
    \end{pmatrix} = \begin{pmatrix}
        29.16 && 21.6\\
        21.6 && 16
    \end{pmatrix}
\end{equation*}

\subsection*{C}
\begin{equation*}
    A = S_w^{-1}S_B
\end{equation*}

\begin{equation*}
    S_w^{-1} = \frac{1}{343.64} \begin{pmatrix}
        26.4 && 2.2 \\ 2.2 && 13.2
    \end{pmatrix} = \begin{pmatrix}
        0.076 && 0.006 \\ 0.006 && 0.038
    \end{pmatrix}
\end{equation*}

\begin{equation*}
    \Rightarrow A = \begin{pmatrix}
        2.34 && 1.73 \\ 0.99 && 0.73
    \end{pmatrix}
\end{equation*}

\begin{equation*}
    \Rightarrow A - \lambda I \begin{pmatrix}
        2.34 - \lambda && 1.73 \\ 0.93 && 0.73 - \lambda
    \end{pmatrix}
\end{equation*}
\begin{equation*}
    |A - \lambda I| = (\lambda - 2.34)\times(\lambda - 0.73) = 0
\end{equation*}

\begin{equation*}
    \Rightarrow \lambda^2 - 3.07\lambda - 0.004 = 0
\end{equation*}
\begin{equation*}
    \lambda_1 = -0.001, \lambda_2 = 3.07
\end{equation*}

Thus the greatest eigen value is \textbf{3.07}.

\section*{Question 2}
TODO

\end{document}